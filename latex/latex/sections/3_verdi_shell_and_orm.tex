% !TEX root = ../AiiDA_tutorial.tex
\section[Verdi shell and AiiDA objects]{Using the verdi shell and familiarizing
with AiiDA objects\label{shell}}

In this section we will use an interactive ipython environment with all the basic AiiDA classes
already loaded. We propose two realizations of such a tool. The first consist of a special ipython shell where all the AiiDA classes, methods and functions are accessible.
Type in the terminal
%Open a second terminal, connect to the Amazon machine with \cmd{ssh} and type: 
\begin{bashcommand}
 verdi shell
\end{bashcommand}
For all the everyday AiiDA-based operations, i.e. creating,
querying and using AiiDA objects, the \cmd{verdi shell} is probably the best tool.
In this case, we suggest that you use two terminals, one for the \cmd{verdi shell} and one to execute bash commands.

The second option is based on \texttt{jupyter} notebooks and is probably most suitable to the purposes of our tutorial. Go to the browser where you have opened \texttt{jupyter} and click \texttt{New} $\to$ \texttt{Python 2} (top right corner). This will open an ipython notebook based on cells where you can type portions of python code. The code will not be executed until you press \verb|Shift+Enter| from within a cell. 
Type in the first cell 
\begin{pythoncommand}
 %aiida
\end{pythoncommand}
and execute it. This will set exactly the same environment as the \cmd{verdi shell}.
The notebook will be automatically saved upon any modification and when you think you are done, you can export your notebook in many formats by going to \verb|File| $\to$ \verb|Download as|. We suggest you to have a look to the drop-down menus \texttt{Insert} and \texttt{Cell} where you will find the main commands to manage the cells of your notebook. 
\textbf{The \cmd{verdi shell} and the \cmd{jupyter} notebook are completely equivalent. Use either according to your personal preference.}

Note: you will still need sometimes to type command-line
instructions in \cmd{bash} in the first terminal you opened today. 
To differentiate these from the commands to be typed in the
\cmd{verdi shell}, the latter will be marked in this document by a vertical line
on the left, like:
\begin{pythoncommand}
 some verdi shell command
\end{pythoncommand}
while command-line instructions in \cmd{bash} to be typed on a terminal will
be encapsulated between horizontal lines:
\begin{bashcommand}
 some bash command
\end{bashcommand}
Alternatively, to avoid changing terminal, you can execute \cmd{bash} commands within the \cmd{verdi shell} or the notebook adding an exclamation mark before the command itself
\begin{pythoncommand}
 !some bash command
\end{pythoncommand}


%%%%%%%%%%%%%%%%%%%%%%%%%%%%%%%
% Load node and .res method
%%%%%%%%%%%%%%%%%%%%%%%%%%%%%%%

\subsection[load_node]{Loading a node\label{load_node}}

Most AiiDA objects are represented by nodes, identified in the database by its pk number (an integer). 
You can access a node using the following command in the shell:
\begin{pythoncommand}
 node = load_node(PK)
\end{pythoncommand}
Load a node using one of the calculation pks visible in the graph you displayed
in the previous section of the tutorial. Then get the energy of the
calculation with the command
\begin{pythoncommand}
 node.res.energy
\end{pythoncommand}
You can also type
\begin{pythoncommand}
 node.res.
\end{pythoncommand}
and then press \cmd{TAB} to see all the possible output
results of the calculation.

%%%%%%%%%%%%%%%%%%%%%%%%%%%%%%%
% Loading several kinds of nodes
%%%%%%%%%%%%%%%%%%%%%%%%%%%%%%%

\subsection{Loading different kinds of nodes}

\subsubsection{Pseudopotentials}

From the graph displayed in Section~\ref{sec:aiida_graph}, find the pk of the barium pseudopotential file (LDA). Load it and verify that it describes barium. Type
\begin{pythoncommand}
 upf = load_node(PK)
 upf.element
\end{pythoncommand}
All methods of \texttt{UpfData} are accessible by typing \texttt{upf.} and then pressing \cmd{TAB}.

\subsubsection{k-points}

A set of k-points in the Brillouin zone is represented by an instance of the
\texttt{KpointsData} class. Choose one from the graph of Section~\ref{sec:aiida_graph},
load it as \texttt{kpoints} and inspect its content:
% kpoints = load_node(PK)
\begin{pythoncommand}
 kpoints.get_kpoints_mesh()
\end{pythoncommand}
Then get the full (explicit) list of k-points belonging to this mesh using
\begin{pythoncommand}
 kpoints.get_kpoints_mesh(print_list=True)
\end{pythoncommand}
If you incurred in a \texttt{AttributeError}, it means that the kpoints instance does not 
represent a regular mesh but rather a list of 
k-points defined by their crystal coordinates (typically used when plotting a band structure).
In this case, get the list of k-points coordinates using
\begin{pythoncommand}
 kpoints.get_kpoints()
\end{pythoncommand}
If you prefer Cartesian (rather than crystal) coordinates, type
\begin{pythoncommand}
 kpoints.get_kpoints(cartesian=True)
\end{pythoncommand}

%%%%%%%%%K points%%%%%%%%%%5
%\subsubsection{K-points definition}
For later use in this tutorial, let us try now to create a kpoints instance, to 
describe a regular $2\times2\times2$ mesh of k-points, centered at the Gamma point (i.e. without offset).
This can be done with the following commands:
\begin{pythoncommand}
 from aiida.orm.data.array.kpoints import KpointsData
 kpoints = KpointsData()
 kpoints_mesh = 2
 kpoints.set_kpoints_mesh([kpoints_mesh,kpoints_mesh,kpoints_mesh])
 kpoints.store()
\end{pythoncommand}

The import performed in the first line is however unpractical as it requires to remember the exact location of the module containing the KpointsData class. Instead, it is easier to use the \cmd{DataFactory} function instead of an explicit import.

\begin{pythoncommand}
 KpointsData = DataFactory("array.kpoints")
\end{pythoncommand}

This function loads the appropriate class defined in a string (here \texttt{array.kpoints}).\footnote{The string provided to the \cmd{DataFactory} encodes both the location and the name of the required class according to some specific rules.}
Therefore, \texttt{KpointsData} is not a class instance, but the kpoints class itself!


\subsubsection{Parameters}

Nested dictionaries with individual parameters, as well as lists and arrays, are
represented in AiiDA with \texttt{Parameter\-Data} objects. Get the PK and load
the input parameters of a calculation in the graph of Section~\ref{sec:aiida_graph}.
Then display its content by typing
\begin{pythoncommand}
 params.get_dict()
\end{pythoncommand}
where \texttt{params} is the \texttt{ParameterData} node you loaded. %You can
%also access directly the first level of dictionary keys by pressing \cmd{TAB}
%after having typed
%\begin{pythoncommand}
% params.dict.
%\end{pythoncommand}
Modify the dictionary content so that the wave-function cutoff is now set to 20 Ry. 
Note that you cannot modify an object already stored in the database.
To save the modification, you must create a new ParameterData object. Similarly to what discussed before, first load the \texttt{ParameterData} class by typing
\begin{pythoncommand}
 ParameterData = DataFactory('parameter')
\end{pythoncommand}
Then an instance of the class (i.e. the parameter object
that we want to create) is created and initialized by the command
\begin{pythoncommand}
 new_params = ParameterData(dict=YOUR_DICT)
\end{pythoncommand}
where \texttt{YOUR\_DICT} is the modified dictionary.
Note that the parameter object is not yet stored in the database.
In fact, if you simply type \texttt{new\_params} in the verdi shell, you will be prompted with a string notifying you the ``unstored'' status.
To save an entry in the database corresponding to the \texttt{new\_params} object, you need to type a last command in the verdi shell:
\begin{pythoncommand}
 new_params.store()
\end{pythoncommand}

\subsubsection{Structures}

Find a structure in the graph of Section~\ref{sec:aiida_graph} and load it. Display
its chemical formula, atomic positions and species using
\begin{pythoncommand}
 structure.get_formula()
 structure.sites
\end{pythoncommand}
where \texttt{structure} is the structure you loaded. If you are familiar with ASE and PYMATGEN,
you can convert this structure to those formats by typing
\begin{pythoncommand}
 structure.get_ase()
 structure.get_pymatgen()
\end{pythoncommand}
%You then have access to all the ASE/PYMATGEN methods.
Let's try now to define a new structure to study, specifically a silicon
crystal.
In the \cmd{verdi shell}, define a cubic unit cell as a $3\times3$ matrix, with lattice parameter $a_{lat}=5.4$ \AA:
\begin{pythoncommand}
 alat = 5.4
 the_cell = [[alat/2,alat/2,0.],[alat/2,0.,alat/2],[0.,alat/2,alat/2]]
\end{pythoncommand}

{\bf Note}: Default units for crystal structure cell and coordinates in AiiDA are \AA.

Structures in AiiDA are instances of \texttt{StructureData} class: load it in the verdi shell
\begin{pythoncommand}
 StructureData = DataFactory("structure")
\end{pythoncommand}
Now, initialize the class instance (i.e.\ is the structure we want to study) by the command
\begin{pythoncommand}
 structure = StructureData(cell=the_cell)
\end{pythoncommand}
which sets the cubic cell defined before. From now on, you can access the cell with the command
\begin{pythoncommand}
 structure.cell
\end{pythoncommand}
Finally, append each of the 2 atoms of the cell command. You can do it
using commands like
\begin{pythoncommand}
 structure.append_atom(position=(alat/4.,alat/4.,alat/4.),symbols="Si")
\end{pythoncommand}
for the first `Si' atom. Repeat it for the other atomic site $\left(0,0,0\right)$.
You can access and inspect\footnote{if you set the structure incorrectly, for example with overlapping atoms, it is very likely that any DFT code will fail!} the structure sites with the command
\begin{pythoncommand}
 structure.sites
\end{pythoncommand}
If you make a mistake, start over from \texttt{structure = StructureData(cell=the\_cell)}, or equivalently use \\
\texttt{structure.clear\_kinds()} to remove all kinds (atomic species) and sites.
Alternatively, AiiDA structures can also be
converted directly from ASE~\cite{ref:ASE} structures using\footnote{We purposefully do not provide advanced commands
for crystal structure manipulation in AiiDA, because python packages that accomplish such tasks
already exist (such as ASE or pymatgen).}
\begin{pythoncommand}	
 from ase.lattice.spacegroup import crystal
 ase_structure = crystal('Si', [(0,0,0)], spacegroup=227,
                 cellpar=[alat, alat, alat, 90, 90, 90],primitive_cell=True)
 structure=StructureData(ase=ase_structure)
\end{pythoncommand}
%
Now you can store the new structure object in the database with the command:
\begin{pythoncommand}
 structure.store()
\end{pythoncommand}
%
%
Finally, we can also import the silicon structure from an external (online) repository
such as the Crystallography Open Database~\cite{ref:COD}:
\begin{pythoncommand}
from aiida.tools.dbimporters.plugins.cod import CodDbImporter 
importer = CodDbImporter()
for entry in importer.query(formula='Si',spacegroup='F d -3 m'):
        structure = entry.get_aiida_structure()
        print "Formula", structure.get_formula()
        print "Unit cell volume: ", structure.get_cell_volume()
\end{pythoncommand}
In that case two duplicate structures are found for Si.
%\footnote{Note: one could compare these two structures with the pymatgen tool
%StructureMatcher}

%%%%%%%%%%%%%%%%%%%%%%%%%%%%%%%
% .inp / .out methods
%%%%%%%%%%%%%%%%%%%%%%%%%%%%%%%

\subsection{Accessing inputs and outputs}

Load again the calculation node used in Section~\ref{load_node}:
\begin{pythoncommand}
 calc = load_node(PK)
\end{pythoncommand}
Then type
\begin{pythoncommand}
 calc.inp.
\end{pythoncommand}
and press \cmd{TAB}: you will see all the link names between the
calculation and its input nodes. You can use a specific linkname to access the
corresponding input node, e.g.:
\begin{pythoncommand}
 calc.inp.structure
\end{pythoncommand}

You can use the \cmd{inp} method multiple times in order to browse the graph. For instance, 
if the input structure node that you just accessed is the output of another calculation, you can access the latter by typing

\begin{pythoncommand}
 calc2 = calc.inp.structure.inp.output_structure
\end{pythoncommand}
Here \texttt{calc2} is the \texttt{PwCalculation} that produced the structure used as an input for \texttt{calc}.

Similarly, if you type:
\begin{pythoncommand}
 calc2.out.
\end{pythoncommand}
and then \cmd{TAB}, you will list all output link names of the calculation.
One of them leads to the structure that was the input of \texttt{calc} we loaded
previously:
\begin{pythoncommand}
 calc2.out.output_structure
\end{pythoncommand}
Note that links have a single name, that was assigned by the
calculation that used the corresponding input or produced the corresponding
output, as illustrated in Fig.~\ref{fig:graph}.

For a more programmatic approach, you can get a list of the inputs and outputs of a node, say \texttt{calc}, with the methods
\begin{pythoncommand}
 calc.get_inputs()
 calc.get_outputs()
\end{pythoncommand}

Alternatively, you can get a dictionary where the keys are the link names and the values are the linked objects, with the methods
\begin{pythoncommand}
 calc.get_inputs_dict()
 calc.get_outputs_dict()
\end{pythoncommand}

Note: You will sometime see entries in the dictionary with names like \texttt{output\_kpoints\_3511}.  These exist because standard python dictionaries require unique key names while link labels may not be unique. Therefore, we use the link label plus the PK separated by underscores.


%%%%%%%%%%%%%%%%%%%%%%%%%%%%%%%%%%%%%
% Upf Families
%%%%%%%%%%%%%%%%%%%%%%%%%%%%%%%%%%%%%

\subsection{Pseudopotential families}

Pseudopotentials in AiiDA are grouped in ``families'' that contain one
single pseudo per element. We will see how to work with UPF pseudopotentials (the format used by Quantum ESPRESSO and some other codes).\\
Download and untar the SSSP~\cite{ref:SSSP} pseudopotentials via the
commands:
% wget http://www.materialscloud.org/sssp/pseudos/SSSP_eff_PBESOL.tar.gz
\begin{bashcommand}
 wget https://archive.materialscloud.org/file/2018.0001/v1/SSSP_efficiency_pseudos.tar.gz
 tar -zxvf SSSP_efficiency_pseudos.tar.gz
\end{bashcommand}
Then you can upload the whole set of pseudopotentials to AiiDA by to the following
\texttt{verdi} command:
\begin{bashcommand}
verdi data upf uploadfamily SSSP_efficiency_pseudos 'SSSP' 'SSSP pseudopotential library'
\end{bashcommand}
In the command above, \texttt{SSSP\_efficiency\_pseudos} is the folder containing the pseudopotentials, 'SSSP' is the name given to the family and the last
argument is its description.\\
Finally, you can list all the pseudo families present in the database with
\begin{bashcommand}
 verdi data upf listfamilies
\end{bashcommand}

